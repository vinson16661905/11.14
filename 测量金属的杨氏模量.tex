% !TeX program = xelatex
\documentclass[11pt,a4paper]{article}
\usepackage{xeCJK}
\usepackage{fontspec}
\usepackage{amsmath,amssymb}
\usepackage{siunitx}
\usepackage{booktabs}
\usepackage{graphicx}
\usepackage{caption}
\usepackage{geometry}
\geometry{left=25mm,right=25mm,top=25mm,bottom=25mm}

\setCJKmainfont{SimSun} % 如果需要可更改为系统中可用的中文字体
\setmainfont{Times New Roman}

\title{测量金属的杨氏模量实验报告}
\author{师翔宇\quad 2500011575}
\date{\today}

\begin{document}
\maketitle

\noindent 实验室给出 $g=\SI{9.801}{\meter\per\second\squared}$

\section{CCD法}

\subsection{质量}
\begin{table}[h!]
\centering
\caption{质量测量(单位:g)}
\begin{tabular}{@{}cccccccccc@{}}
\toprule
i & 1 & 2 & 3 & 4 & 5 & 6 & 7 & 8 & 9 \\ \midrule
$m/$g & 200.06 & 199.68 & 199.74 & 200.15 & 200.39 & 200.01 & 200.08 & 200.05 & 200.14 \\ \bottomrule
\end{tabular}
\end{table}

$m_0=\SI{0.05}{g}$,\quad $e=\SI{0.005}{g}$

\[
\sigma_{\text{电子天平}}=\SI{0.00288675}{g},\qquad \sigma_{\text{avrg}}=\SI{0.214126}{g}
\]

\[
 m \pm \sigma_m = \SI{199.98 \pm 0.21}{g}
\]

\subsection{直径}
\begin{table}[h!]
\centering
\caption{直径测量(单位:mm)}
\begin{tabular}{@{}ccccccccccc@{}}
\toprule
i & 1 & 2 & 3 & 4 & 5 & 6 & 7 & 8 & 9 & 10 \\ \midrule
$D/$mm & 0.329 & 0.322 & 0.325 & 0.324 & 0.325 & 0.326 & 0.326 & 0.329 & 0.330 & 0.328 \\ \bottomrule
\end{tabular}
\end{table}

$D_0=\SI{0.005}{mm}$,\quad $e=\SI{0.004}{mm}$

\[
\sigma_{\text{千分尺}}=\SI{0.0023094}{mm},\qquad \sigma_{\text{avrg}}=\SI{0.00254733}{mm}
\]

\[
D \pm \sigma_D = \SI{0.321 \pm 0.003}{mm}
\]

\subsection{金属丝长度}
\[
L=\SI{78.00}{cm},\qquad e=\SI{0.15}{cm}
\]
\[
L \pm \sigma_L = \SI{78.00 \pm 0.09}{cm}
\]

\subsection{伸长量}
\begin{table}[h!]
\centering
\caption{CCD测量的伸长量(单位:mm)}
\begin{tabular}{@{}cccc@{}}
\toprule
i & $m$/g & $r_i$/mm & $r_i'$/mm & $\bar r_i$/mm \\ \midrule
0 & 0.00   & 2.03 & 2.00 & 2.015 \\
1 & 200.01 & 2.21 & 2.29 & 2.250 \\
2 & 399.64 & 2.42 & 2.40 & 2.410 \\
3 & 599.33 & 2.58 & 2.55 & 2.565 \\
4 & 799.43 & 2.63 & 2.65 & 2.640 \\
5 & 999.77 & 2.82 & 2.82 & 2.820 \\
6 & 1199.73 & 2.98 & 2.96 & 2.970 \\
7 & 1399.76 & 3.08 & 3.10 & 3.090 \\
8 & 1599.76 & 3.20 & 3.20 & 3.200 \\
9 & 1799.85 & 3.31 & 3.31 & 3.310 \\ \bottomrule
\end{tabular}
\end{table}

将最小分度值 $\SI{0.05}{mm}$ 视为允差,得到
\[
\sigma_{\text{CCD}}=\SI{0.0288675}{mm}.
\]

\subsection{逐差法数据处理}
注意到 $i=0$ 的数据线性不佳,可能是因为金属丝有弯折,故舍去,从 $i=1$ 的数据开始处理。

\begin{table}[h!]
\centering
\caption{逐差法数据片段(单位:mm)}
\begin{tabular}{@{}cccc@{}}
\toprule
i & 1 & 2 & 3 & 4 \\ \midrule
$\delta\bar r_i$ & 0.570 & 0.560 & 0.525 & 0.560 \\
$\delta L_i$ & 0.1425 & 0.14 & 0.13125 & 0.14 \\ \bottomrule
\end{tabular}
\end{table}

\[
\sigma_{\text{avrg}}=\SI{0.00493447}{mm},\qquad 
\sigma_{\delta L}=\sqrt{\left(\frac{\sigma_{\text{CCD}}}{5}\right)^2+\sigma_{\text{avrg}}^2}=\SI{0.00759489}{mm}.
\]

\[
\delta L \pm \sigma_{\delta L} = \SI{0.138 \pm 0.008}{mm}.
\]

根据公式:
\[
E=\frac{4 m g L}{\pi d^2 \delta L} = 1.3689\times 10^{11}\ \mathrm{Pa}.
\]

不确定度:
\[
\sigma_E = E\sqrt{\left(\frac{\sigma_m}{m}\right)^2+\left(\frac{\sigma_L}{L}\right)^2+\left(\frac{2\sigma_d}{d}\right)^2+\left(\frac{\sigma_{\delta L}}{\delta L}\right)^2}
=0.083407\times 10^{11}\ \mathrm{Pa}.
\]

\[
E \pm \sigma_E = (1.37 \pm 0.08)\times 10^{11}\ \mathrm{Pa}.
\]

\subsection{最小二乘法处理数据}
\begin{figure}[h!]
\centering
\includegraphics[width=0.6\textwidth]{CCD.png}
\caption{CCD法图像}
\end{figure}

图像解析式:
\[
r = \frac{mgL}{\pi E d^2} + L,
\]
斜率 $k=\dfrac{gL}{\pi E d^2} = 6.6578\times 10^{-4}\ \mathrm{mm/g} = 6.6578\times 10^{-4}\ \mathrm{m/kg}.$

由此得到
\[
E=\frac{4 g L}{\pi k d^2}=1.41884\times 10^{11}\ \mathrm{Pa}.
\]

\[
\sigma_{k,A}=1.65463\times 10^{-5}\ \mathrm{m/kg},\qquad
\sigma_{k,B}=\frac{e/\sqrt{3}}{\sqrt{\sum_1^9 (m_i-\bar m)^2}}=1.86325\times 10^{-5}\ \mathrm{m/kg},\
\sigma_k=2.49189\times 10^{-5}\ \mathrm{m/kg}.
\]

\[
\sigma_E=E\sqrt{\left(\frac{\sigma_L}{L}\right)^2+\left(\frac{2\sigma_d}{d}\right)^2+\left(\frac{\sigma_k}{k}\right)^2}
=0.0593811\times 10^{11}\ \mathrm{Pa}.
\]

\[
E \pm \sigma_E = (1.42 \pm 0.06)\times 10^{11}\ \mathrm{Pa}.
\]

\section{光杠杆法}

\subsection{质量}
\begin{table}[h!]
\centering
\caption{质量测量(光杠杆法,单位:g)}
\begin{tabular}{@{}ccccccccccc@{}}
\toprule
i & 1 & 2 & 3 & 4 & 5 & 6 & 7 & 8 & 9 & 10 & 11 \\ \midrule
$m/$g & 199.55 & 199.83 & 199.73 & 199.82 & 200.07 & 199.88 & 199.96 & 199.80 & 199.98 & 199.90 & 200.07 \\ \bottomrule
\end{tabular}
\end{table}

$m_0=\SI{0.00}{g}$,\quad $m=\SI{199.872}{g}$.

\subsection{直径}
\begin{table}[h!]
\centering
\caption{直径测量(单位:mm)}
\begin{tabular}{@{}cccccccccc@{}}
\toprule
i & 1 & 2 & 3 & 4 & 5 & 6 & 7 & 8 & 9 & 10 \\ \midrule
$D/$mm & 0.329 & 0.322 & 0.325 & 0.324 & 0.325 & 0.326 & 0.326 & 0.329 & 0.330 & 0.328 \\ \bottomrule
\end{tabular}
\end{table}

$d_0=\SI{0.005}{mm}$,\quad $d=\SI{0.3221}{mm}$.

\subsection{仪器参数}
\[
L=\SI{71.10}{cm},\quad D=\SI{8.80}{cm},\quad R=\SI{134.0}{cm}.
\]

\subsection{伸长量}
\begin{table}[h!]
\centering
\caption{光杠杆法伸长量(单位:cm)}
\begin{tabular}{@{}ccccc@{}}
\toprule
i & $m$/g & $r_i$/cm & $r_i'$/cm & $\bar r_i$/cm \\ \midrule
0 & 0.00 & 2.94 & 2.97 & 2.955 \\
1 & 199.55 & 2.60 & 2.61 & 2.605 \\
2 & 399.38 & 2.26 & 2.26 & 2.260 \\
3 & 599.11 & 1.91 & 1.93 & 1.920 \\
4 & 798.93 & 1.60 & 1.59 & 1.595 \\
5 & 999.00 & 1.26 & 1.26 & 1.260 \\
6 & 1198.88 & 0.96 & 0.96 & 0.960 \\
7 & 1398.84 & 0.61 & 0.62 & 0.615 \\
8 & 1598.64 & 0.30 & 0.29 & 0.295 \\
9 & 1798.62 & 0.01 & 0.00 & 0.005 \\
10 & 1998.52 & -0.30 & -0.30 & -0.300 \\
11 & 2198.59 & -0.61 & -0.61 & -0.610 \\ \bottomrule
\end{tabular}
\end{table}

\subsection{逐差法处理}
\begin{table}[h!]
\centering
\caption{逐差法(光杠杆)}
\begin{tabular}{@{}ccccccc@{}}
\toprule
i & 1 & 2 & 3 & 4 & 5 & 6 \\ \midrule
$l$ & -1.995 & -1.990 & -1.965 & -1.915 & -1.895 & -1.870 \\ \bottomrule
\end{tabular}
\end{table}

$l=\SI{0.323056}{cm}$

依据公式:
\[
E=\frac{8 m g L R}{\pi d^2 D l} = \SI{1.61e11}{Pa}.
\]

\subsection{最小二乘法处理}
\begin{figure}[h!]
\centering
\includegraphics[width=0.6\textwidth]{光杠杆.png}
\caption{光杠杆法图像}
\end{figure}

图像解析式:
\[
l = l_0 + \frac{8 g L R}{\pi d^2 D E} m,
\]因此
\[
E=\frac{8 g L R}{\pi d^2 D k} = \SI{1.61e11}{Pa}.
\]

\section{金属梁法}

注:由于时间原因,直接使用实验一已知质量的前6个砝码。

\subsection{有效长度}
\[
l=\SI{25.00}{cm}.
\]

\subsection{宽度与厚度}
\begin{table}[h!]
\centering
\caption{宽度 a(单位:mm)}
\begin{tabular}{@{}ccc@{}}
\toprule
i & 1 & 2 & 3 \\ \midrule
$a/$mm & 10.02 & 10.08 & 10.12 \\ \bottomrule
\end{tabular}
\end{table}
\[
a=\SI{10.0733}{mm}.
\]

\begin{table}[h!]
\centering
\caption{厚度 h(单位:mm)}
\begin{tabular}{@{}ccccc@{}}
\toprule
i & 1 & 2 & 3 & 4 & 5 \\ \midrule
$h/$mm & 1.652 & 1.661 & 1.620 & 1.688 & 1.619 \\ \bottomrule
\end{tabular}
\end{table}
\[
h=\SI{1.648}{mm}.
\]

\subsection{挠度}
\begin{table}[h!]
\centering
\caption{挠度 $\lambda$(单位:mm)}
\begin{tabular}{@{}ccc@{}}
\toprule
i & $m$/g & $\lambda$/mm \\ \midrule
0 & 0.00 & 43.885 \\
1 & 200.01 & 42.927 \\
2 & 399.64 & 41.912 \\
3 & 599.33 & 40.967 \\
4 & 799.43 & 39.978 \\
5 & 999.77 & 38.882 \\
6 & 1199.73 & 37.950 \\ \bottomrule
\end{tabular}
\end{table}

\subsection{逐差法数据处理}
\begin{table}[h!]
\centering
\caption{逐差法(挠度,单位:mm)}
\begin{tabular}{@{}ccc@{}}
\toprule
i & 1 & 2 & 3 \\ \midrule
$\delta\lambda$/mm & 3.907 & 4.045 & 3.962 \\ \bottomrule
\end{tabular}
\end{table}

\[
\lambda=\SI{0.992833}{mm}.
\]

由公式:
\[
E=\frac{m g l^3}{4 \lambda a h^3} = \SI{1.71e11}{Pa}.
\]

\subsection{最小二乘法处理}
\begin{figure}[h!]
\centering
\includegraphics[width=0.6\textwidth]{金属梁.png}
\caption{金属梁法图像}
\end{figure}

图像解析式:
\[
\lambda=\lambda_0+\frac{g l^3}{4 a E h^3} m,
\]从拟合系数 $k$ 得到
\[
E=\frac{g l^3}{4 a k h^3} = \SI{1.71e11}{Pa}.
\]

\bigskip
\noindent 完成。
\end{document}